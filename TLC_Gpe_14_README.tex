\documentclass{report}
\usepackage[utf8]{inputenc}
\usepackage[T1]{fontenc}
\usepackage[french]{babel}
\usepackage{graphicx}
\usepackage{amsmath}
\usepackage{amssymb}
\usepackage{listings}             % Include the listings-package
%%reduction espace avant titre
\usepackage{titling}
\setlength{\droptitle}{-3cm}
%%reduction des marges
\usepackage{geometry}
\geometry{hmargin=2.5cm,vmargin=1.5cm}
\makeatletter
%%chapitre sur une ligne
\def\@makechapterhead#1{%
  \vspace*{50\p@}%
  {\parindent \z@ \raggedright \normalfont
    \interlinepenalty\@M
   \Huge \bfseries\thechapter.\quad#1\par\nobreak
   \vskip 20\p@
  }}
\makeatother
\title{
	\noindent\hrulefill\\
	\vspace{0.4\baselineskip}	%%saut de 0.4 ligne
	Compte-rendu de projet\\
	{\large Théorie des langages et compilation}\\
	\noindent\hrulefill
	\vspace{2\baselineskip}
}
\author{
	\textit{Auteur(s) :}\\
	Romain CHARPENTIER\\
	\textit{romain.charpentier@etu.univ-poitiers.fr}\\\\
	Lucille MOISE\\
	\textit{lucille.moise@etu.univ-poitiers.fr}\\\\\\
}
\date{\today}

\begin{document}
\lstset{language=C++}
	\begin{figure}
	\vspace{5\baselineskip}
   	 \begin{minipage}[c]{.3\linewidth}
   	     \centering
   	     \includegraphics[scale=0.5]{../../../Images/universite.png} 
 	   \end{minipage}
  	  \hfill
  	  \begin{minipage}[c]{.3\linewidth}
   	     \centering
    	    \includegraphics[scale=0.5]{../../../Images/sfa.jpg}
    	\end{minipage}
	\end{figure}
	\maketitle
	\newpage
	\section{Description des fonctionnalités}
	Pendant la réalisation de notre projet, nous avons codé un interpréteur de langage objet. Nous avons divisé le code qui est interprété en plusieurs instructions : les déclarations de variable, les déclarations de classe, les affectations et les appels de méthode. A la fin de la reconnaissance d'une instruction par l'interpréteur, celui-ci va visiter l'instruction à travers l'objet "Interpretor" qui se chargera d'exécuter l'instruction. Celui-ci va également communiquer avec une table des symboles (symbolTable) qui se chargera de sauvegarder les déclarations.\\
	\par
	Il est ainsi possible de déclarer des variables de types primitifs et des variables objet. Les types primitifs sont boolean, float et integer. Les objets sont définis par une déclaration de classe qui est réalisée avant la déclaration dudit objet. Lors de la visite de la déclaration, l'objet Interpretor va créer des variables correspondant aux déclarations dans la table des symboles. Elles n'auront pour l'instant aucune valeur mais il est possible d'en mettre une.
	\begin{lstlisting}
		exemple is integer; //declaration d'integer
		exemple2 is exemple3; //declaration d'un objet
	\end{lstlisting}
	\par
	Les classes comportent 2 emplacements data et method qui vont respectivement contenir les déclarations des variables et des méthodes. Les 2 emplacements peuvent être vides. Lors de la création d'un objet, ce dernier va avoir un pointeur vers sa classe pour accéder aux méthodes et va créer des variables pour chaque déclaration de variable dans la classe. Donc chaque objet aura ses propres attributs mais partagera les mêmes méthodes avec les autres objets de même classe.
	\begin{lstlisting}
		class exemple3 is
		data
		//declaration des attributs
		method
		//declaration des methodes
		end exemple3;
	\end{lstlisting}
	\par 
	L'emplacement method des classes contient donc des fonctions. Les fonctions peuvent être de 2 types : VoidFonction et ReturnFonction. Le premier correspond aux fonctions correspondant à une instructione et le second correspond aux fonctions renvoyant une expression (return). Les méthodes peuvent avoir également plusieurs paramètres de n'importe quel type. Et il est également possible de réaliser des affectations multiples dans les méthodes. Dans l'exemple, la method2 initialise les variables a et b à la valeur i.
	\begin{lstlisting}
		class exemple3 is
		data
			a is integer;
			b is integer;
		method
			method1() is return 1;	//ReturnFonction
			method2(i : integer) is (a,b):=(i,i); //VoidFonction
		end exemple3;
	\end{lstlisting}
	Dans le programme principal, les méthodes de cette classe pourront donc être appelées grâce à la classe Call qui contiendra toutes les informations relatives à l'appel. Ensuite l'interpréteur cherchera l'objet dans la table des symboles et à partir de celui-ci on pourra accéder à la méthode de sa classe et donc exécuter l'instruction.
	\section{Choix de structuration}
	Pour la structuration, la grammaire va lire instruction par instruction comme détaillé précédemment. Les classes seront donc regroupés en plusieurs classes "majeures" Classe, Expression et Instruction. Les autres classes sont des spécificités de celles-ci et héritent donc de ces dernières. 
	\section{Jeux d'essais}
	\begin{table}[!h]
		\begin{center}
		\begin{tabular}{|c|c|}
			\hline
			\textbf{Nom du fichier} & \textbf{Fonctionnalité illustrée}\\
			\hline
			fichier 1 & fonctionnalité\\
			\hline
			fichier 1 & fonctionnalité\\
			\hline
			fichier 1 & fonctionnalité\\
			\hline
			fichier 1 & fonctionnalité\\
			\hline
		\end{tabular}
		\end{center}
		\caption{Description des jeux d'essais}
	\end{table}
\end{document}
	